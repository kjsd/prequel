\section*{在ストックホルム日本公使館館内。}

駐在外交官・菅田が、栗野と秋月のふたりを迎える。
\vskip.5\baselineskip
菅田	「ついに始まりましたな。戦争が」

秋月	「大国ロシアとの大戦争です。アジア人がはじめて行なう、欧米人との本格戦争。見せてや
  りましょう、大日本帝国の底力を!」
\vskip.5\baselineskip

菅田、ため息をつく。
\vskip.5\baselineskip

菅田	「日本は、とてもロシアのような大国と戦争をして勝てるような国じゃない。

  日本は早まったことをしでかした」頭を抱える。

栗野	「戦争はもう始まったのだ。われわれは国民の半分を殺してでもロシアに勝たねばならぬ」

菅田	「せめて伊藤先生が言っていたようにロシア皇帝と同盟を取り結んでいれば……」

秋月	「よくはないです」
\vskip.5\baselineskip
明石、コートの肩に雪を乗せて入ってくる。
\vskip.5\baselineskip
秋月	「弱小国が、ロシアと同盟を組んだところで、そのあとに待っているのは強迫外交だけだ。
  すこしでも皇帝の機嫌を損ねれば、一国の首都だろうと戦車で踏み込んでくる。見てください。ス
  ウェーデンも、フィンランドも、これではまるでロシアの奴隷国だ」

\vskip.5\baselineskip
明石、室内でコートをバダバタやる。小雪が飛ぶ。
\vskip.5\baselineskip
秋月	「こっちに来る前にポーランドを見てきましたが、あそこもひどい。許せませんよ」
\vskip.5\baselineskip
明石、ストーブにケツを寄せる。ケツから湯気がたつ。

栗野、眉をひそめる。
\vskip.5\baselineskip
栗野	「明石君、当地には石鹸というものがある」

明石	「は?」

栗野	「明石君。君、臭うぞ。セッケンを使うことだ。日本の外交官たるもの常に颯爽として、気
  品を持たねばならん」
\vskip.5\baselineskip
明石、くんくんとシャツを匂う。
\vskip.5\baselineskip
××××

路傍で自分のマフラーの匂いをかぐイリューシャ

××××
\vskip.5\baselineskip
明石、ニヤニヤする。
\vskip.5\baselineskip
明石	「臭いませんが」

栗野	「明石君。君は語学は堪能だが、見た目がむさくるしくていかん」

明石	「セッケンでセンシャに勝てますかね」
