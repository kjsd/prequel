\section*{河沿いのカフェ。}

明石、ずるずるに着崩れた洋装。

宇都宮、ぴったりに仕立てられた洋装。立派な英国紳士ぶり。

ふたり、路上端のテーブルに座っている。
\vskip.5\baselineskip
宇都宮「やあ、あいかわらずいい狸顔だ。ほっとする顔だな」

明石「ほっといてください」

宇都宮「干されて拗ねてるそうじゃないか」

明石「本営のエライ人には好まれてませんもので……。拗ねちゃないですよ」

宇都宮「エライひと……児玉サンか」

明石「軍隊じゃ見栄えのいい乃木サンみたいなのが可愛がられるんです。どうせ、ぼくみたいな狸面じゃ、率いられる兵士も死にきれんでしょうよ」
\vskip.5\baselineskip
宇都宮、笑う。

明石、むっとする。
\vskip.5\baselineskip
明石「外務省のエライさんが、ぼくなんかになんのお話があるんですか?」

宇都宮「ロンドンは、世界の情報が集まる。あそこにいれば、クニにいる君の嫁はんが、いつ誰と寝たのかも、すぐにわかるぞ」

明石「知りたくないですね。戦局はどうです?」

宇都宮「勝ち目ナシだな。日本はもう兵隊もカネもカラッケツだが、ロシアは左うちわだ。シベリア鉄道が完成したら今の3倍の兵力がピストン輸送されるようになるだろう」

明石「新聞が言ってることと違いますね。フランスでは、日本が快進撃してるって聞きましたよ。この戦争はこのまま日本が勝つって」

宇都宮「ロシア軍の意図的後退だ。満州にいる日本軍の補給路をしたたか延長させて、一気に包囲逆襲してくる。奉天が危ない。ロシア軍が反撃に出た時が日本軍の全滅するときだ」

明石「ちくしょう思った通りだ。帝国陸軍は猪突猛進で脇が甘い。見てくれ華々しいから、ヨーロッパの記者連中には、日本の快進撃に見えるんでしょうね」

宇都宮「その観測は丙だな。君は、報道てものがわかってない」

明石「なんですか、記者連中がおもしろがって快進撃報道にしてる、とでも?」

宇都宮「記者じゃない。日英同盟のなせる技さ。世界各地の情報は一度ロンドンにかき集められ、ロンドンから世界へ回覧される」

明石「それがなにか?」

宇都宮「日本は自力で戦争ができるほど金持ちじゃない。列強からカネを借りて戦争をしている。負けそうな国には、だれもカネを貸しちゃくれん。カネがなければ、砲弾も作れず戦えん」

明石「そういうことを国民が知らないじゃないですか。日本の新聞も新聞だ。外国報道を真に受けて、どうどう自力でロシアをやっつけられるかのようなことを書くから、国民だって勘違いする。火の車になってる帳簿公開して、丸々借金だって言えばいい。日本は外国と戦争できるほど金持ちの国じゃないって」

宇都宮「だから、君はそこがわかってない。そんな内情を知って喜ぶのは、ロシア軍だ。日本の困窮を知れば、さらに嵩にかかって日本に踏み込んでくる」

明石「敵に隠し事をするために、自国民にも目隠しをしてしまうんですか。それで戦争遂行は国民の総意だなんて、よく言える。いつかろくなことにならなくなりますよ」

宇都宮「地球は極楽浄土じゃないんだよ。他にどうすべきか策があるなら言ってみたまえ」
\vskip.5\baselineskip
明石、即座の返答に困る。
\vskip.5\baselineskip
宇都宮「まあ、聞け。ロイター通信は、弱小可憐な日本に同情的だ。連中にとっちゃ、これは童話の中の戦争なんだ。ヨーロッパ最強帝国の悪の皇帝が、農民と漁民しかいない小さな小さな平和な島を略奪しようとしている。しかして勇敢にも皇帝の大軍に立ち向かう善玉日本人。勇者ここにあり」
\vskip.5\baselineskip
明石、鼻で笑う。
\vskip.5\baselineskip
明石「ヨーロッパの諸国家は、もとからロシアの侵略本能に危機感を持っているのです。いいように代用されているだけだ。なにが勇者だ」

宇都宮「ち、い、さ、な、勇者だ。こういう言葉が大事なんだよ。諸国の同情を引かなければ日本は侵略から身を護れん」
\vskip.5\baselineskip
明石、頭を抱える。
\vskip.5\baselineskip
明石「日本の大本営は……」

宇都宮「他国に侵略されるくらいなら、国民がひとりのこらず死に絶えるまで、戦うことを命じるだろうぜ」
\vskip.5\baselineskip
▽
\vskip.5\baselineskip
宇都宮「私は夕方の船便でロンドンに戻る」
\vskip.5\baselineskip
宇都宮、懐中時計を見る。
\vskip.5\baselineskip
「ところで、君いまいくら持ってる?」

明石「カネ? ここの支払いなら持ちますよ」

宇都宮「そんなハシタガネじゃない。国が買えるくらいのカネ」

明石「大金なら、栗野公使が、百万円持ってますね」

宇都宮「すごい。百万とは破格値だな。」

明石「公使館の運用資金です。外交機密費てやつ」

宇都宮「栗野は、それをどう使うつもりなんだ?」

明石「和平工作でしょう。……ペテルブルグの宮廷貴族にばらまいて、皇帝に日本との和平説得をさせようとしています」

宇都宮「それはダメだ。それは開戦前に伊藤博文もやろうとしたが失敗してる。もっとも、それで良かったんだが、……ロシアは公約を守る国じゃない。あんな国と不可侵同盟条約なんか結んだヒにゃあ、いいようにテゴメにされて捨てられるだけだ。スウェーデンがいい例じゃないか。日本はイギリスと組めたのが最大の幸運だったんだ。これを使わないテはない」

明石「ははあ、東欧の栗野公使に百万円が落ちて、ロンドンのあなたには落ちなかったてあたりで、拗ねてんですね」

宇都宮「私は、干されてはないぞ。地球の裏側で、日本のために忙しく働いているよ。言ってるだろう。ロンドンには世界中の情報があつまり、そこから……」

明石「カネとか報道とかなんかじゃ戦争には勝てません。それもガセ報道なんて。戦争は実弾でするものです」

宇都宮「君、栗野さんの百万円、掠めとれんか?」

明石「なに言ってんですか?」

宇都宮「カネと報道で戦争に勝つ」立ち上がる

宇都宮「東欧は君に任せる」

明石「任せる? なにを? ぼくは陸軍大佐です。文官の横槍をなあなあと聞いてちゃ軍規が崩壊します」

宇都宮「……。児玉サンには、プランパワーがあるよ。君を呼び戻さないのは考えあってのことだと思う。私には、なんとなくわかる」

明石「だから。同郷だからってつるまれては困ります。大将も大将だ。軍政と国政をごっちゃにするような、ああいう・・」
\vskip.5\baselineskip
宇都宮、聞いていない、欧米人のように背中を向けたまま手を振ってちゃきちゃき店をでてゆく。
\vskip.5\baselineskip
明石「……なにができるってんだ、今の俺に」
