\section*{在ストックホルムアメリカ大使館。}

各国の大使、夫人が社交会を開いている。

栗野公使、燕尾服で入ってくる。

壁際にいたトルージン、栗野公使の前に立つ。
\vskip.5\baselineskip
トルージン	「ドーブルゥィ、公使クリノ。勝ち目のない戦争だと気づいて、停戦仲介をしてくれる国を探しに来たのかな?」

栗野	「ヴェーチェル。それも正当な外交任務ですよ、署長」

トルージン	「正当。正当とはよいことだ。貴君とは敵対国の人間だが、互いにルールを踏み越えなければ、こうして機嫌よく話し合うこともできる」

栗野	「高等警察が国際法に則った正しい逮捕権を理解されておることを切に願いますな」
\vskip.5\baselineskip
栗野、挑発的な眼でデオルキンを見る。
\vskip.5\baselineskip
トルージン	「ハラショー、クリノ。東洋のサルにスパイゲームは向かない事をはっきりと思い知らせてやる」

栗野	「バジャールスタ」
\vskip.5\baselineskip
窓際に明石と秋月。
\vskip.5\baselineskip
秋月	「堂々としたものです。栗野公使」

明石	「日本は、あいつらサムライが自分で自分らの支配権を捨てて革命をした。だから今でも上から民衆をみてる、その態度が抜けないんだ」

秋月	「毅然としてていいじゃないですか。節度もあって、欧州の騎士階級に通じるものがある。サムライは、好評ですよ」

明石	「ずっとサムライならいいかもしれん。だが、ああいう態度だけは、代々下の者にも伝染する。日本の官僚はこの先ろくなものにならなんだろうよ」
\vskip.5\baselineskip
秋月、揺らせていたグラスをしみじみ見る。
\vskip.5\baselineskip
秋月	「国と国が利害を競って、戦いつづけている限り・・。敵国というものがあり、敵の目があるかぎり、国家の方策を国民にも知らせたり相談したりすることはできません。一部エリートが国策を独占し懸念し構想し、真実ではない理由を国民に喧伝して国を引っ張る他ないじゃないですか」
\vskip.5\baselineskip
明石、手に持っていたシャンパングラスを手近なテーブルに置く。
\vskip.5\baselineskip
明石	「君は若い。そのエリートだ。だからそんな無理なことが当り前だと信じれるんだな」

秋月	「道理ですよ」
\vskip.5\baselineskip
明石、自分のグラスと他人のグラスの位置を勝手に動かしている。
\vskip.5\baselineskip
明石	「……。たとえば、通信をだよ」
\vskip.5\baselineskip
明石、グラスの配列をあれこれ調整して、心地いい光景を探っている。
\vskip.5\baselineskip
明石	「一部団体やら国家やらから切り離すことができたら、その道理は無理になるだろうな」

秋月	「明石さんは夢想家ですな」笑う。
\vskip.5\baselineskip
明石、会心の作が出来ましたな如く、グラスの並んだテーブルを眺める。

給仕が、グラスを片付ける。
\vskip.5\baselineskip
明石	「北欧貴夫人とお近づきになれんのかのお。金髪碧眼、でへは」

秋月	「その為にここへ連れてけって言ったンですかあ?」

明石	「ぼかぁ、スパイだよ。スパイに美女はつきものだろうに」

秋月	「サムライ日本は礼儀正しいいい子の集まりなんですからね。停戦交渉のいい仲介国を得るには、各国大使にいい心証を与えなくてはならない。くれぐれもバカしないでくださいよ」

明石	「言われて、ハイそうですかと聞くと思うか俺が」
\vskip.5\baselineskip
明石、宴遊会場をぶらぶら歩き出す。酔っぱらっていて妙に機嫌がいい。

大きな丸い中華テーブルを囲む列強諸外国の外交官、武官がいる。

英・仏・外交官が座って談笑し、独・露・武官が、立ち話ししている。

明石、テーブルに近づきテーブルの食べ物を物色する。
\vskip.5\baselineskip
英外交官	「(諸君、となりの小さなテーブルから新しいお客さんだ)」
\vskip.5\baselineskip
一同、明石を見る。
\vskip.5\baselineskip
イギリス	「(ようこそ、我々の中華テーブルへ)」

明石	「あはいあはい」輪に押し入り料理に手を出す。

仏外交官	「(本当にどこにでも突っ込んでくるんだな)」明石に微笑む

露武官	「(機関砲の前に整列したままぞろぞろやってくる。知能のなさ、空恐ろしいくらいでね)」笑う

独武官	「(あなたは、ドイツ語かフランス語ができますか?)」

明石	「あはいあはい」前歯を出してニヤニヤ笑い。

露武官	「(ほっとけ、わかってない)」

イギリス	「(さて諸君。このように彼らは、このテーブルに分け入ってきたわけだが)」

フランス	「(堂々、清国を戦争で打ち負かして、だね?)」
\vskip.5\baselineskip
英外交官のかたわらに英国貴婦人。チャイナドレスを着ている。
\vskip.5\baselineskip
貴婦人	「(黙って見過すつもり?)」

イギリス	「(彼らは、戦争に勝利したのだから、領土を受け取るのは、当然の権利だろう)」
\vskip.5\baselineskip
一同、明石が中華料理を取るのを眺める。
\vskip.5\baselineskip
ロシア	「(それは人間同士の約束事だ)」

フランス	「(ご婦人に)」明石の取った皿を貴婦人に渡すように手で示す。
\vskip.5\baselineskip
明石、わからなげに首をかしげる。
\vskip.5\baselineskip
ロシア	「(だれか、社交ルールを教えてやってくれないかね)」
\vskip.5\baselineskip
露武官、料理を手盛りした明石から皿を取り上げ貴婦人に渡す。
\vskip.5\baselineskip
明石	「あはいあはい」チャイナドレスの美女に微笑みかける。
\vskip.5\baselineskip
仏外交官、貴婦人が受け取った皿から料理を取り、皿を独武官に回す。
\vskip.5\baselineskip
ドイツ	「(ここは、我々に入用なのだ)」笑いながら、料理を取り、露武官に回す。
\vskip.5\baselineskip
露武官、料理をさらい、空になった皿を明石に返す。
\vskip.5\baselineskip
明石	「あはいあはい」空の皿に料理を盛ろうとする。
\vskip.5\baselineskip
露武官、巨体で明石を押し出す。目線は、独武官を牽制している。

明石、さすがにむっとする。
\vskip.5\baselineskip
フランス	冷笑「(君たちは必死ダネ)」
\vskip.5\baselineskip
露・独・武官、英仏を睨む。
\vskip.5\baselineskip
ドイツ	「(君らは、早くからこのテーブルに着いていたから、さぞかし満腹だろうが)」

ロシア	「(我々だって、まだたいして食っとらんぞ!)」

貴婦人	「(なんだか怖いわ)」

フランス	「(マダム心配には及びません、私がついています。ドイツ人よイタリア人よロシア人よ、そこのサルよ、乱暴はよしたまえ)」

ドイツ	「(それだ。すこし出遅れて来ただけで、ことあるごとに我々のことを餓えた狼のごとく、言いふらすのは、得心いかないな)」

ロシア	「(そうとも。君らがさんざんやってきたことではないか)」

イギリス	「(粘着はよしましょう)」

イギリス	「(我々は紳士だし、もう20世紀になったのです)」
\vskip.5\baselineskip
英外交官、ナプキンで口元をぬぐう。
\vskip.5\baselineskip
ロシア	「(フランスはわがロシアと同盟関係にある。いつからイギリス寄りにかわったのだ? イギリスは、このサルなどと組みしおったのに)」明石を指差す。

フランス	「(友よ。ロシアが極東に注力して欧州を空にしてしまっている今、私としてもイギリスに向かって強きに出るわけにはいかないではないか)」

イギリス	「(このままロシア軍が敗退してくれると、私には好都合ですな)仏大使に酒を注ぐ。

フランス	「(私はね、ロシアの軍事力を頼りに同盟を組んだのに、その肝心の軍が極東で猿などを相手にやり込められ続けられては困るのですよ。私の仇敵は、あくまでもアナタなのだから)」英外交官に返杯し微笑む。

ロシア	「(陰謀だ。我々は、勝っている!)」

フランス	「(なあ友よ。猿の島などどうでもよいではないか。君らがあまり小突き回すから、彼らは発狂したのではないのかね)」

ロシア	「(それも誤解だ。我々は、猿の島などに関心はない。我々が握っておきたいのは極東。中国、朝鮮の一部にすぎん)」

フランス	「(そいつに聞いてみてはどうか?)」明石を指差す「(日本はな、防衛戦争のつもりでいる。侵略者から自国を守っていると思っている。彼らには、正義の戦争を行なっているという強い意思がある)」

ロシア	「(猿どもに、そう信じ込ませた奴がいるのだ)」英外交官を睨む。

フランス	「(ロシアにその意図がないとしても。ロシア軍が朝鮮半島に居座れば、小さな海を挟んで向かい合わせだ。日本人でなくとも、侵略を警戒するだろう)」

イギリス	「(ロシア軍が、朝鮮から進軍を始めれば、猿は自分の島で防衛戦をしなくてはならなくなる。戦略的に言って、そうなってからでは、日本本土の防衛は絶対に不可能だ)」

フランス	「(だから、そうなる前に満州で戦うことにしたのだろう)」

貴婦人	「(迷惑ですわ)」

ドイツ	「(仕向けた、のだろう?)」英大使を斜に見る。

英外交官	「(レトリックも対話のひとつですよ。彼らがどう解釈したかは知らないが)」

貴婦人	「(こんな話、私はもううんざり)」

フランス	「(そうですな。マダム、踊りましょう)」
\vskip.5\baselineskip
チャイナドレスの貴婦人、仏外交官の手をいなして、テーブルを離れる。

英外交官、指を弾く。

米大使館付きの若いアメリカ人給仕長、皿を片付ける。
\vskip.5\baselineskip
アメリカ給仕長、厨房へさがり、料理人たちを叱咤し、取りまとめ、新しい料理をふんだんに用意する。フロアに戻って来て、全体の様子を眺める。
\vskip.5\baselineskip
控えめに宴席を運営している者独特の機敏さと余裕があり、今は、その挙措が、慎ましい。
\vskip.5\baselineskip
仏外交官、貴婦人にあしらわれて、席に戻ってくる。
\vskip.5\baselineskip
フランス	「(正直、潮時ではないかと思っているのだが)」

イギリス	「(調停役を買ってでられますか?うーん、それは見合わせてもらいたいな)」

フランス	「(アナタは、露日戦争が早期に終わって、ロシア軍が欧州に戻って欲しくないからそう言うのでしょう)」

イギリス	「(そうですよ。舞い戻られては困る。誤解がないように言っておきますが、イギリスが日本に肩入れするのは、日本単独ではとうていロシア軍と戦えないからだ。私は、露日、双方に共倒れしていただきたいのだ。皇帝ロシア軍は欧州の安全にとって、実に脅威だし、アジアに雄があってもらっても不都合だ。そこはみなさん同意見だと思いますがね)」仏・独を見る。
\vskip.5\baselineskip
露武官、酔っ払って目が据わってくる。
\vskip.5\baselineskip
ロシア	「(猿を駆逐して、我々は大軍のまま欧州に舞い戻る!)」

イギリス	「(なにか作戦がおありのようですね?)」

ロシア	「(なぜ、教えねばならんのだ?)」

イギリス	「(阻止不能で確実な計画なら隠し立てする必要はない。むしろ喧伝された方が、より優位に事を運べますよ。自信がおありなのでしょう。で、いつ?)」

ロシア	「(この冬だ。猿どもは、冬に軍隊は動けないと信じている。冬こそロシア軍の本領だ)」

ドイツ	「(細部ではなく、全体の運動としても、日本軍は突出しすぎいてる。補給の機構というものがない。側面を衝けば一撃だろう)」
\vskip.5\baselineskip
露、明石の開いたわき腹をしげしげ眺める。
\vskip.5\baselineskip
ロシア	「(それだけの兵力輸送は完了した。奉天で日本陸軍に対する包囲殲滅戦を行ない、猿どもを屠殺する)」

フランス	「(言い切ったね)」

ドイツ	「(ロシアと日本は、大陸満州で陸戦をしている。ロシアは陸続きだが、日本が自分の陸軍を支えるには、本土から海を渡って武器弾薬食料を運ばなければならない。補給路を確保するには、制海権を守りつづけなければならない。なのにいまだに、旅順は陥落していない)」

フランス	「(旅順?)」フロアの貴婦人たちの踊りに合わせて軽く腰を振り、あくびをかみ殺す。

ドイツ	「(旅順には、ロシア艦隊がある。バルト海から回航されたロシア本国の艦隊と合同すれば、日本海軍の2倍以上の大艦隊になる)」
\vskip.5\baselineskip
独武官、テーブルにグラスを置く。そのひとつのグラスの左右にふたつのグラスを置いて挟み撃ちにして見せる。
\vskip.5\baselineskip
ドイツ	「(艦隊戦というのは不沈戦艦が備えた大砲の数で決まる。日本海軍に勝ち目はない)」

ロシア	「(陸海ともども、動員数で圧倒している)」

ドイツ	「(日本海の制海権はロシアが握り、大陸に展開する日本陸軍は、補給路を断たれ孤立し、取り残され、冬の包囲戦で、全滅するだろう)」

フランス	「(それじゃあ、いま調停に入るのは馬鹿げてるな)」笑って英外交官の肩を叩く。

ロシア	「(調停? とんでもない。そういうものは、戦争をしている国同士にあるものだ。我々は戦争をしているのではない。コイツらを駆除しているのだ)」  明石を指差して微笑む。
\vskip.5\baselineskip
明石、ニヤーと笑って応える。
\vskip.5\baselineskip
イギリス	「(どうも私までが形勢危しだな)」仏・独・露を見まわす。

ドイツ	「(イタリアと相談しあってみてはどうだ?)」
\vskip.5\baselineskip
一同、どっと笑う。
\vskip.5\baselineskip
フランス	「(ルーブル美術館に浮世絵コレクションが増やせそうかね?)」

ロシア	「(猿の曲芸画などいくらでもお譲りする)」

明石	「あはいあはいあはい」ニヤニヤ笑いながら、その場の全員と握手し、立ち去る。
\vskip.5\baselineskip
仏外交官、フィンガーボールで手を洗う。

英外交官、席を立ち、米大使に挨拶しに向かう。
\vskip.5\baselineskip
明石、秋月の所にもどってくる。
\vskip.5\baselineskip
明石	「いい子にしてたろ」

秋月	「ペコペコにやにやして、なんとも、絵にならない人だなあ」

明石	「エル・ポデルトタール・エッサ・エンセス・マヌス」

秋月	「何語です?」

明石	「ポルトガル語だ」

秋月	「ほう、で、意味は?」
\vskip.5\baselineskip
明石、ドアを蹴っ飛ばして、でてゆく。
