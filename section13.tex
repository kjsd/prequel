\section*{陸軍省館内、高等警察本部。}

早朝。

トルージンとザーノフ、中庭に列なした警官たちを見下ろす。
\vskip.5\baselineskip
トルージン「皇帝陛下の慈悲深きこと。しかし、猿にはそのありがたさがわからない。猿ゆえ戦力の相違が数えられず、いたずらに手向かいおるのだ」

ザーノフ「困った猿どもですな」

トルージン「猿などはよい。だが、猿の狂騒に興じて踊る愚かな人間が、このヨーロッパにも現れる」

ザーノフ「人間には理性があります。猿と一緒にされては……」

トルージン「踏み潰したフィンランド。その反抗残党が、このストックホルムの地下に潜んでおるとの情報もある。人間とはいえ、いわゆる、狂信者、どもだな。こういった狂人どもが、猿の狂騒を猿真似んとも限らん」

ザーノフ「わがロシア高等警察の任務は、その狂人どもをいぶりだし全員の息の根を止めることにあるわけですな」
\vskip.5\baselineskip
ザーノフ、心なしか窓に寄り、トルージンの視線から中庭を塞いでいる。

トルージン、押しのけて、警官の列を見下ろす。列に穴が目立つ。
\vskip.5\baselineskip
トルージン「なぜこうも怠惰な連中なのだ」

ザーノフ「しかし、本国の選りすぐりです。高等警察には平民などは一人もおらず……」

トルージン「狂信者どもと猿を結び付けねばそれでよい! 数の少ない猿をよく観察しろ」
\vskip.5\baselineskip
トルージン、日本公使館職員の名簿と職歴書をみる。
\vskip.5\baselineskip
トルージン「とくに、老いた猿。素面な人間をさらに選りすぐって、こいつをがっちり監視させておけ」
\vskip.5\baselineskip
トルージン、公使栗野の書類をザーノフにつき渡す。
