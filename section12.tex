\section*{在ストックホルム日本公使館。}

栗野、菅田、明石、秋月、がフォークとナイフで夕食をとっている。

食材は貧困だが、明石を除き、他の者は、それぞれの理由で、マナーはいい。

滑稽なほど、過剰に良い。
\vskip.5\baselineskip
菅田「そして今、ロシア皇帝の欲望は極東の小さな日本に向いています。皇帝の欲するものは太平洋に面した軍港なのです。」

秋月「渡しません」フルーツを曲芸なみの器用さで分解する。

秋月「帝国陸軍が、満州でロシア軍の進軍を阻止します。迎撃し追い返す」
\vskip.5\baselineskip
菅田、気弱気に、明石の方を見る。
\vskip.5\baselineskip
明石「ロシア軍の動員兵力は日本の十倍以上だ。満州に展開した日本軍は、踏みつぶされる」
\vskip.5\baselineskip
秋月、鼻で笑う。
\vskip.5\baselineskip
秋月「それが、帝国陸軍明石大佐の見分ですか?」

明石、頷く。
\vskip.5\baselineskip
秋月「日本人には、大和魂があります。露助が何人束になろうと、気迫横溢したひとりの日本兵にはかなわない」

明石「文官がそんな馬鹿なことを口走ってもらっちゃ困るんだが」

秋月「大陸清国をさえ軽々とやっつけた帝国陸海軍ではないですか。明治開闢以来、列強毛唐どもの不当な態度に、国民の堪忍袋の緒は限界にきてるんです。世界を我が物と考え欲しいままに奪う列強の傲慢さ、これに鉄槌を下すは、神州日本の使命です。国民は大いに発揚しています。大佐、国民のオーダーですよ。帝国陸海軍は、国民の期待に応えて当然でしょう」

明石「無理なもんは無理なんだ」
\vskip.5\baselineskip
秋月、ため息をつく。
\vskip.5\baselineskip
秋月「なにを根拠に。明石さん、新聞読んでますか? ロイターは日本軍の優勢を報じてますよ。世界がそう言ってます」口を拭いたナプキンを膝に捨てる。
\vskip.5\baselineskip
明石、首をひねる。
\vskip.5\baselineskip
明石「なぜ、だろう?」心底わからなさ気。

秋月「皇国の兵は、みな潔く戦って死ぬ覚悟だ。なによりこれが強い」

明石「潔く、か」

栗野「明石君。靴が焦げている。あれは君のだろう?」
\vskip.5\baselineskip
栗野、ストーブの上で湯気をあげている靴を指差す。

秋月、鼻を押さえる。

明石、手近にある布で手を拭き、のそのそ立ち上がって、ストーブに近づき、靴を取る。
\vskip.5\baselineskip
秋月「明石さん。年上のあなたにこんなこと言いたくはないのですが、あなたは言うことやること、ちょっと常識に欠けてやしませんか?」
明石「ほう、常識?」

秋月「天皇陛下に命を捧げることは日本人としてごく当り前の常識です。あなたも一人前の大人なら、そのあたりのことをきっちりわきまえるべきだ」

明石「そうだね」靴の乾き具合を確かめている。

秋月「軍務研究もよいですが、もうすこし、世間を見られてはどうですか。どうも、あなたはズ
  レていてやりづらい」
