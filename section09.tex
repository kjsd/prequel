\section*{在ストックホルム日本公使館。}

明石と菅田。

菅田、窓辺に立っている。横顔が、憔悴しきっている。

明石、靴を乗せたストーブに足を向けている。
\vskip.5\baselineskip
明石	「こちらは長いのですか?」

菅田	「ええ、文明、というものをさんざん見ましたよ」
\vskip.5\baselineskip
明石、素足の指を曲げたり伸ばしたりしている。
\vskip.5\baselineskip
明石	「畑を耕し漁をする我々とは、ずいぶんちがいますね。産業革命。これは地球を変えてしまった。起こった以上は乗らなければならない。乗り遅れたものが奴隷にされる」

菅田	「明石さん。あなたは、ロシア本国に駐在したこともある武官だ。露都ペテルブルグの日本大使館が引き引き揚げになった時、どうして帰国しなかったのです? いま帰らなかったら・・・次はないかもしれない」

明石	「日本がなくなっているかも?」
\vskip.5\baselineskip
菅田、ぼんやり窓の外の凍った光景を見詰める。
\vskip.5\baselineskip
菅田	「ご存知でしょう。ロシアの占領政策の容赦のなさを。負ければ人の棲む国とは扱われない。自主憲法は取り上げられ、国会議事堂は高等警察の本部にでもなるでしょう。虎ノ門あたりは政治犯を収容する監獄街になる。そこでは連日、銃殺刑が執行されるでしょう。東大寺も日光東照宮も潰され、替わりにロシア正教の、巨大なタマネギ頭をした大聖堂が建つ。対馬は要塞島になる。横須賀はロシア太平洋艦隊の基地に。日本人は、日本人であることを禁止され……」

明石	「……妻はロシア語でおかえりというのだろうな……」

菅田	「ああ、そういえば」
\vskip.5\baselineskip
菅田、執務机の抽斗をあけ、外交官郵袋の中から手紙を取り出して明石に渡す。
\vskip.5\baselineskip
菅田	「あなたに個人郵便が届いてましたよ。奥様ですかな」

明石	「ちがうようです」手紙をポケットに突っ込む。

菅田	「日本はそろそろ梅、ですかな」
\vskip.5\baselineskip
明石、ポケットから手紙を取り出して封を切る。

明石、菅田、なんとなく鼻から空気を吸う。
\vskip.5\baselineskip
明石	「末の弟ですよ。金沢で会社勤めを」文面を見ている「……徴集がきて、兵役につくようだ」

明石	「第九師団。に決まったようです。苦労の九ですかね」

菅田	「楽曲でいえば、第九、はめでたい響きじゃないですか。武運あれかしです」
\vskip.5\baselineskip
明石、窓の外の凍った風景をなんとなく見る。
