\section*{在ストックホルム日本公使舘。栗野執務室。}

執務机に公使栗野。

その前に、明石。
\vskip.5\baselineskip
明石	「ロンドンの宇都宮さんに電信打ってもらえませんかね」

栗野	「どんな?」

明石	「ロシアに革命の兆しアリ」

栗野	「ナイだろ」卓上の茶碗を触り温度を確かめている。

明石	「いや! アリます」

栗野	「あるにはあるのだろうが、運動としては、まだ小さい」

明石	「だから、ソコを大々的にセンセーショナルに、もう今日にも明日にも、ロシア全土に革命の炎が上がって、ペテルブルグ宮廷は、いつなんどき暴徒に夜襲されるかわかったものではない。てなガセ報道をロンドンから世界に向かって発信してもらいたいンです」

栗野	「面倒だな」
\vskip.5\baselineskip
栗野、茶碗が熱かったのか、指を引っ込める。熱すぎる茶に不機嫌になる。当番は誰か、こんな茶の煎れかたは言語道断。てなシブい顔。
\vskip.5\baselineskip
明石	「いいですか? イギリスのロイター通信と言えば、世界最大の報道機関だ。ロイターが正義と言えば正義、悪と言えば悪。事実と言えば、なんだって事実になるんです」

栗野	「それは誤報をしないからだろう」

明石	「誤報はすでにしてますよ。日本が快進撃してるなんてとんでもないウソだ。大日本帝国が可憐なよい子の国だというのも誤報だ。誤報というより情報操作だ。でも、そのおかげで日本は世界から信用され、カネも貸してもらえて戦争ができてる。事実が報道に引きずられてるんです」

栗野	「極論だな。報道より世間を見ろ」
\vskip.5\baselineskip
栗野、椅子の背もたれに体重をかけ、窓の外を見る。茶が冷めるのを待つ風情。
\vskip.5\baselineskip
明石	「知ってますか? 日本の参謀本部もロシア軍もロイター通信の情報をもとに作戦を立てている。ペテルブルグ宮廷の貴族連中だって、世界の状況はロイターから吸収してるんです。軍のスパイうんぬんと民間の報道機関、考えてくださいよ、どっちが人数も多くて、気軽に情報を採りやすいと思いますか? まじめに考えて」

栗野	「それで? それが、この戦争と、どういう、関係が、あるのかね」
\vskip.5\baselineskip
栗野、茶碗の蓋をそっとあげて、茶柱が立ってるや否やをよく確かめてから、優雅にすする。
\vskip.5\baselineskip
明石	「……かつてないことです。かつてないことですが、かつてない卑劣な戦略でロシアを攻撃できます」

栗野	「大袈裟に言うことはない。おばはんの井戸端会議で村八分にしようというのだろう? 世界に冠たる大ロシアの皇室を」

明石	「そう! それ! それです!」

栗野	「下劣」
\vskip.5\baselineskip
栗野、茶を一服し、年代物のテーブルに気を使い、仏語の並んだ報告書の一枚を取って茶碗の下に敷く。
\vskip.5\baselineskip
栗野	「士道に悖るふるまいだ」
\vskip.5\baselineskip
栗野、のどの潤いを愉しむかのよう。
\vskip.5\baselineskip
明石	「富を掠めるなら貿易、……リョーマが衝いた通りだ。そして操作するなら報道……。公使! 5年前から、20世紀になったんですよ。ご存知なかったんですか?」
\vskip.5\baselineskip
栗野、黙って前かがみになる。冷静なようでいて、異常な速度で立ち上がる。
\vskip.5\baselineskip
栗野	「知っチョーわ!」顔を真っ赤にして怒っている。机を叩く。

栗野	「維新を作ったのはわしらじゃけん。おまえら洟垂れ小僧が」
\vskip.5\baselineskip
明石、食う。
\vskip.5\baselineskip
明石	「そのご自慢の維新! 公使は純真一路、汚いことはなにひとつなくやり遂げたんですね!」
\vskip.5\baselineskip
栗野が叩いた弾みで転んだスロバキア製の地厚な陶器の湯呑が欧州骨董執務机の縁をめざしてころころ転がってゆく。
\vskip.5\baselineskip
栗野	「……」

明石	「あんたぁ!」
\vskip.5\baselineskip
明石、姿勢も変えない。平静な風にゆったりつっ立ったままで、そのまま怒鳴っている。
\vskip.5\baselineskip
明石	「呆けたとではナカトですか?」
\vskip.5\baselineskip
栗野、ニガイ顔。

極東。列島。西日本。北九州。ハカタ言葉とフクオカ言葉は近在でも異なる。

実にふたりは同郷。
\vskip.5\baselineskip
栗野	「外国の大物にも、それだけの大口を利く度胸が、あるかキサン」

明石	「誰相手でも、態度を変えたことはないですよ。知ってるでしょう」

栗野	「口八丁で真剣勝負か、なさけなか」

明石	「カタナよりは有効なものだと悟りました」
\vskip.5\baselineskip
栗野、縁から落ちそうな湯のみを掴む。
\vskip.5\baselineskip
明石	「自分は、戦争をしとるとです。ガイコクと戦争してるんです」

栗野	「宇都宮君にそれを頼むと、どうなるのかね」椅子に座る

明石	「ロシアの民衆も自分たちの不満が世界に通用するイラダチだという事を自覚するでしょう。反逆行為に自信も持つでしょう。その自信が、事実革命を引き
起こす弾みになるかもしれません。ロシア皇帝は、逆に恐怖にかられます。貴族たちもです。気ぞらしの対外戦争どころではなくなる」

栗野	「それで、極東にいるロシア軍の進行が止まる、とでも言うのか」
\vskip.5\baselineskip
栗野、湯呑をきちんと自分の正面に置く。
\vskip.5\baselineskip
明石	「そこまでは、なんとも言えません。でも……戦争は、始めるより終らせることの方が難しい。この戦争、もうこの辺でヤめとこうとおもえる状況てなんです?」
\vskip.5\baselineskip
栗野、置いた湯呑を観賞している。
\vskip.5\baselineskip
明石	「日本軍が、地球の半分以上もある広いユーラシア大陸をどこまでも、攻めあがり、シベリアを越えて進撃し、露都ペテルブルグを焼き尽くすまでやりますか?」
\vskip.5\baselineskip
栗野、湯呑をちょっと動かし、机にかかる針葉樹の陰に埋めて眺め直す。
\vskip.5\baselineskip
明石	「できるわけない」
\vskip.5\baselineskip
栗野、気にいらなかったように、陶器を木立ちの陽と影の狭間に置き直している。
\vskip.5\baselineskip
明石	「それよりも、日本の国民が半分も死んで、日本本土が占領される方が先でしょう。そこまでやったらロシア皇帝も陛下も満足するでしょう。……ソレ、ほんとにソレでいいんですか?」

栗野	「……」机の引き出しから、金属工具を取り出す。レンチ。

明石	「そこへ行き着く前に、皇帝に、ここらでヤめとこうと思わせる状況を作らないといけないんじゃナイですか?」

栗野	「根拠のない煽りだけで、革命が起きると、本気で思うのかね」

明石	「根拠はあります。ロシアは、叩けばホコリの出る国です。叩くもなにも、あっちこっちもうホコリだらけでホコリが国になっているようなもんだ。ちょっと火花を飛ばせば、あっという間に燃え広がる。必要なのは、砲弾じゃない。言葉なんです」
\vskip.5\baselineskip
栗野、六角ネジを締めるレンチで湯呑を割る。

学者のように、破片を見分している。
\vskip.5\baselineskip
栗野	「……。ロンドンが、うんというとはかぎらんぞ」
\vskip.5\baselineskip
栗野、木陰が陰影を作る机の上に、なにごとを創作しているのか、破片を丁寧に配置している。
\vskip.5\baselineskip
明石	「栗野さん……」
\vskip.5\baselineskip
栗野、陶器破片の配置された枯山水風の卓上を見下ろしている。
\vskip.5\baselineskip
栗野	「わかった」
\vskip.5\baselineskip
栗野、頷くと、破片を片付け、今しがたまで見下ろしていた光景をパッとゴミ箱に捨てる。
\vskip.5\baselineskip
栗野	「やってみよう。非常識だが、おもしろい。わりと論理的だしな」
