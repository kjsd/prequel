\section*{海洋}

(テキスト表示)

『中世の終焉。ユーラシア大陸北西部に起こった産業革命は、その地域の国々を飛躍的に先進化させ
  た。
  
狩猟、採集、牧畜、農耕など、<地域自給>を主体とした人間が生息する為の穏やかな営み、は、<
工業システム>の獲得によって、変革した。

その変革は、<類人猿>が道具を手にしたことで<人類>に変貌した瞬間以外には他に比べ得る現象
がないほど、極端で、激しい。

例えば、地球の環境データは、その(歴史時間的)瞬間、に、あらゆるグラフが急角度で折れ曲がり、
危険な速度で急加速を続けている。

変革を肯定し、以前の生計体系を破壊してしまったからには、もはや、変革以前には、戻れない。人
類は、人類の生き方を変えた。

産業文明には呼吸がある。

安い材料に満ち溢れ、安い人材を乗せた広大な土地、集団、を渇望する。

加工した産業製品を売りさばく広大な市場を要求する。

この呼吸を失うと、加工者に過ぎない産業文明は、ただちに窒息死する。

近代<帝国主義>は、産業文明が呼吸し続けること、それによって国をして優位たらしめること、を命題として、生まれた。

産業文明によって先進した一部の国々・帝国群が、古代中世的な多数の後進国から各段の優越をなしとげ、産業文明の呼吸が望むままに、掠奪し、地球の大半を手中に収めた。

帝国主義が侵略主義とほぼ同意になるのは、産業文明が持つ呼吸の生理ゆえであり、同じ生理を持つ国家、民族集団間には、必然的に利害摩擦が生じる。もともとは地域間のものだった<戦争>が産業文明の呼吸の望む所以によって、地球規模に拡大した。

産業文明の呼吸、がこれを欲するかきり国家政治・ありとあらゆる人間集団、がどう、体制や名称を改めようと、<帝国>の呼称・体制をどう改めようと、本質の原理は変わらない。それによって起きる様々な現象、悲喜劇、はなにも変わることはない。
\vskip.5\baselineskip
地球規模―国家―からすべての大中小集団―我々―の日常生活の、瑣末な仔細のことごとく、に至るまで。
わたしたちは、この非情なゲーム設定の上で、\underline{いま}、を呼吸し、\underline{ここ}で、
生きている。』

